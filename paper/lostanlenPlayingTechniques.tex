% -----------------------------------------------
% Template for ISMIR Papers
% 2018 version, based on previous ISMIR templates

% Requirements :
% * 6+n page length maximum
% * 4MB maximum file size
% * Copyright note must appear in the bottom left corner of first page
% * Clearer statement about citing own work in anonymized submission
% (see conference website for additional details)
% -----------------------------------------------

\documentclass{article}
\usepackage{ismir,amsmath,cite,url}
\usepackage{graphicx}
\usepackage{color}
\usepackage{xspace}


% Title.
% ------
\title{Extended playing techniques: \\
the next frontier in musical instrument recognition}

% Note: Please do NOT use \thanks or a \footnote in any of the author markup

% Three addresses
% --------------
\threeauthors
  {First Author} {Affiliation1 \\ {\tt author1@ismir.edu}}
  {Second Author} {\bf Retain these fake authors in\\\bf submission to preserve the formatting}
  {Third Author} {Affiliation3 \\ {\tt author3@ismir.edu}}

%% To make customize author list in Creative Common license, uncomment and customize the next line
%  \def\authorname{First Author, Second Author, Third Author}

% Four or more addresses
% OR alternative format for large number of co-authors
% ------------
%\multauthor
%{First author$^1$ \hspace{1cm} Second author$^1$ \hspace{1cm} Third author$^2$} { \bfseries{Fourth author$^3$ \hspace{1cm} Fifth author$^2$ \hspace{1cm} Sixth author$^1$}\\
%  $^1$ Department of Computer Science, University , Country\\
%$^2$ International Laboratories, City, Country\\
%$^3$  Company, Address\\
%{\tt\small CorrespondenceAuthor@ismir.edu, PossibleOtherAuthor@ismir.edu}
%}
%\def\authorname{First author, Second author, Third author, Fourth author, Fifth author, Sixth author}


\sloppy % please retain sloppy command for improved formatting

\hyphenation{ma-rim-ba}
\newcommand*{\eg}{e.g.\@\xspace}
\newcommand*{\ie}{i.e.\@\xspace}
\newcommand*{\resp}{resp.\@\xspace}

\begin{document}

%
\maketitle

%%%%%%%%%%%%%%%%%%%%%%%%%%%%%%%%%%%%%%%%%%%%%%%%%%%%%%%%%%%%%%%%%%%%%%%%%%%%%%%%
%%%%%%%%%%%%%%%%%%%%%%%%%%%%%%%%%% ABSTRACT %%%%%%%%%%%%%%%%%%%%%%%%%%%%%%%%%%%%
\begin{abstract}
Although the automatic recognition of a musical instrument from the recording of a single ``ordinary'' note is close to becoming solved problem, the ability of a computer to precisely identify instrumental playing techniques (IPT) within an extended taxonomy remains far below human accuracy.
This article provides the first benchmark of machine listening systems for the classification of instrumental playing techniques, including  in the symphonic orchestra.
We identify three necessary conditions for significantly outperforming the classical mel-frequency cepstral coefficients (MFCC) baseline:
(1) the inclusion of second-order scattering coefficients to account for the presence of amplitude modulations ;
(2) the inclusion of large temporal scales ; and
(3) the resort to supervised metric learning.
\end{abstract}

%%%%%%%%%%%%%%%%%%%%%%%%%%%%%%%%%%%%%%%%%%%%%%%%%%%%%%%%%%%%%%%%%%%%%%%%%%%%%%%%
%%%%%%%%%%%%%%%%%%%%%%%%%%%%%%%% INTRODUCTION %%%%%%%%%%%%%%%%%%%%%%%%%%%%%%%%%%
\section{Introduction}\label{sec:introduction}
The progressive diversification of the timbral palette in Western classical music at the turn of the 20th century is reflected in five concurrent trends:
the addition of new instruments to the Western symphonic instrumentarium, either by technological inventions (\eg theremin) or importation from non-Western musical cultures (\eg marimba) \cite{sachs2012book};
the creation of novel instrumental associations, as epitomized by \emph{Klangfarbenmelodie} \cite{schoenberg2010book};
the temporary alteration of resonant properties through mutes and other ``preparations'' \cite{dianova2007phd};
a more systematic usage of extended instrumental techniques, such as artificial harmonics, \emph{col legno batutto}, or flutter tonguing \cite{kostka2016book};
and the resort to digital audio effects \cite{zolzer2011dafx}.
The first of these trends has somewhat stalled: to this day, most Western composers rely on an acoustic instrumentarium that is only marginally different from the one that was available in the Late Romantic period.
Nevertheless, the latter approaches to timbral diversification were massively adopted into post-war contemporary music.
In particular, an increased concern for the concept of musical gesture \cite{godoy2009book} has liberated many unconventional instrumental techniques from their figurativistic connotations, thus making the so-called ``ordinary'' playing style merely one of many compositional options.


Far from being exclusive to erudite music, extended playing techniques are also commonly found in oral tradition; in some cases, they even stand out as a distinctive component of musical style.
Four well-known examples are:
the snap pizzicato (``slap") of the upright bass in rockabilly,
the growl of the tenor saxophone in rock'n'roll,
the shuffle stroke of the violin (``fiddle'') in Irish folklore,
and the glissando of the clarinet in Klezmer music.
Consequently, the mere knowledge of organology (the instrumental \emph{what?}~of music), as opposed to chironomics (its gestural \emph{how?}), is a rather weak source of information for browsing and recommending music within large audio databases.

Yet, past research in music information retrieval (MIR), and especially machine listening, too rarely acknowledges the benefits of integrating the influence of performer gestures into a coherent taxonomy of musical instrument sounds.
Instead, gestures are either framed as a spurious form of intra-class variability between instruments, without delving into its interdependencies with pitch and intensity;
or, symmetrically, as a probe for the acoustical study of a given instrument, without enough emphasis onto the broader picture of orchestral diversity.

One major cause of this gap in research is the difficulty of collecting and annotating data for contemporary instrumental techniques.
Fortunately, such obstacle has recently been overcome, owing to the creation of databases of instrumental samples in a perspective of spectralist music orchestration \cite{maresz2013cmr}.
In this article, we capitalize on the availability of data to formulate a new line of research in MIR, namely the joint retrieval of organological information (``\emph{what} instrument is being played in this recordings?'') and chironomical information (``\emph{how} is the musician producing sound?''), while remaining invariant to other factors of variability, which are deliberately regarded as contextual: where, when, why, by whom, and for whom was the music (in this recording) played.

Section 2 reviews the existing literature on the topic of retrieving information from instrumental playing techniques (IPT).
Section 3 derives the task of IPT classification from the definition of both a taxonomy of instruments and a taxonomy of gestures.
Section 4 describes how two topics in machine listening, namely scattering transforms and unsupervised metric learning, are relevant to address this task.
Section 5 reports the results from an IPT classification benchmark on the Studio On Line (SOL) dataset.


%%%%%%%%%%%%%%%%%%%%%%%%%%%%%%%%%%%%%%%%%%%%%%%%%%%%%%%%%%%%%%%%%%%%%%%%%%%%%%%%
%%%%%%%%%%%%%%%%%%%%%%%%%%%%%%%%  RELATED WORK  %%%%%%%%%%%%%%%%%%%%%%%%%%%%%%%%
\section{Related work}

Timbre classification of a single musical instrument (clarinet): \cite{loureiro2004ismir}.
Retrieval of percussion gestures using timbre classification techniques: \cite{tindale2004ismir}.
Polyphonic instrument recognition using spectral clustering: \cite{martins2007ismir}.
% Ref DCASE typology of acoustic events
Knowledge representation issues in musical instrument ontology design: \cite{kolozali2011ismir}.
Guitar playing technique classification: \cite{su2014ismir}.
MedleyDB: \cite{bittner2014ismir}.
Audio Set: \cite{gemmeke2017icassp}.
Visipedia: \cite{belongie2015pattern}
Scattering transforms in musical instrument recognition: \cite{tjoa2010ismir,lostanlen2017phd}.


%%%%%%%%%%%%%%%%%%%%%%%%%%%%%%%%%%%%%%%%%%%%%%%%%%%%%%%%%%%%%%%%%%%%%%%%%%%%%%%%
%%%%%%%%%%%%%%%%%%%%%%%%%%%%%%%%%%%% TASKS %%%%%%%%%%%%%%%%%%%%%%%%%%%%%%%%%%%%%
\section{Tasks}



% Out of scope are:
% Variations in articulation: trill, slide. Unlike vibrato, they have a melodic function.
% Variations in: artificial harmonics, subharmonics [ref Mari Kimura], 
% Variations in phrasing: arpeggio, 

%%%%%%%%%%%%%%%%%%%%%%%%%%%%%%%%%%%%%%%%%%%%%%%%%%%%%%%%%%%%%%%%%%%%%%%%%%%%%%%%
%%%%%%%%%%%%%%%%%%%%%%%%%%%%%%%%%%  METHODS  %%%%%%%%%%%%%%%%%%%%%%%%%%%%%%%%%%%
\section{Methods}
In this section, we point out the theoretical limitations of mel-frequency cepstral coefficients (MFCC) in the representation of musical sounds comprising with extended instrumental playing techniques (IPT), and describe how both the scattering transform and unsupervised metric learning may overcome such limitations.

\subsection{Limitations of mel-frequency cepstral coefficients}

\subsection{Scattering transform}

\section{Large-margin nearest neighbors}


%%%%%%%%%%%%%%%%%%%%%%%%%%%%%%%%%%%%%%%%%%%%%%%%%%%%%%%%%%%%%%%%%%%%%%%%%%%%%%%%
%%%%%%%%%%%%%%%%%%%%%%%%%%%% EXPERIMENTAL RESULTS %%%%%%%%%%%%%%%%%%%%%%%%%%%%%%
\section{Experimental results}


%%%%%%%%%%%%%%%%%%%%%%%%%%%%%%%%%%%%%%%%%%%%%%%%%%%%%%%%%%%%%%%%%%%%%%%%%%%%%%%%
%%%%%%%%%%%%%%%%%%%%%%%%%%%%%%%%%  CONCLUSION  %%%%%%%%%%%%%%%%%%%%%%%%%%%%%%%%%
\section{Conclusion}
% Every quest for information is also a quest for invariance.



\section{Acknowledgments}
% The authors wish to thank Andrew Farnsworth and (?) for fruitful discussions on Visipedia.



% For bibtex users:
\bibliography{lostanlenPlayingTechniques}

\end{document}
