% -----------------------------------------------
% Template for ISMIR Papers
% 2018 version, based on previous ISMIR templates

% Requirements :
% * 6+n page length maximum
% * 4MB maximum file size
% * Copyright note must appear in the bottom left corner of first page
% * Clearer statement about citing own work in anonymized submission
% (see conference website for additional details)
% -----------------------------------------------

\documentclass{article}
\usepackage{ismir,amsmath,cite,url}
\usepackage{graphicx}
\usepackage{color}


% Title.
% ------
\title{Extended playing techniques: \\
the next frontier in musical instrument recognition}

% Note: Please do NOT use \thanks or a \footnote in any of the author markup

% Three addresses
% --------------
\threeauthors
  {First Author} {Affiliation1 \\ {\tt author1@ismir.edu}}
  {Second Author} {\bf Retain these fake authors in\\\bf submission to preserve the formatting}
  {Third Author} {Affiliation3 \\ {\tt author3@ismir.edu}}

%% To make customize author list in Creative Common license, uncomment and customize the next line
%  \def\authorname{First Author, Second Author, Third Author}

% Four or more addresses
% OR alternative format for large number of co-authors
% ------------
%\multauthor
%{First author$^1$ \hspace{1cm} Second author$^1$ \hspace{1cm} Third author$^2$} { \bfseries{Fourth author$^3$ \hspace{1cm} Fifth author$^2$ \hspace{1cm} Sixth author$^1$}\\
%  $^1$ Department of Computer Science, University , Country\\
%$^2$ International Laboratories, City, Country\\
%$^3$  Company, Address\\
%{\tt\small CorrespondenceAuthor@ismir.edu, PossibleOtherAuthor@ismir.edu}
%}
%\def\authorname{First author, Second author, Third author, Fourth author, Fifth author, Sixth author}


\sloppy % please retain sloppy command for improved formatting

\hyphenation{ma-rim-ba}

\begin{document}

%
\maketitle
%
\begin{abstract}

\end{abstract}
%
\section{Introduction}\label{sec:introduction}

The progressive diversification of the timbral palette in Western classical music at the turn of the 20th century is reflected in four concurrent trends:
the addition of new instruments to the Western symphonic instrumentarium, either by technological inventions (e.g. theremin) or importation from non-Western musical cultures (e.g. marimba) ; 
the resort to novel instrumental associations, as epitomized by \emph{Klangfarbenmelodie} ;
the temporary alteration of 
and a more systematic usage of extended instrumental techniques, such as snap pizzicato, col legno batutto, or flutter tonguing.
The first of these trends has now stalled: to this day, most Western composers rely on an acoustic instrumentarium that is only marginally different from the one that was available in the Late Romantic period.
%whereas few others took the radical stance of crafting their own acoustic instruments -- Harry Partch being a noteworthy exception.
Nevertheless, the latter two approaches to timbral diversification, namely polyphonic mixtures and extended instrumental techniques, were massively adopted into post-war contemporary music.



Far from being exclusive to Western classical music, extended playing techniques are also commonly found in the oral tradition.
In some cases, they even stand out as a distinctive component of musical style.
Five well-known examples are:
the snap pizzicato (``slap") in upright bass,
the growling tenor saxophone in rock'n'roll,
the shuffle stroke in Irish fiddle,
and the clarinet glissando in Klezmer music.

% MUSICAL INSTRUMENT RECOGNITION USING BIOLOGICALLY INSPIRED FILTERING OF TEMPORAL DICTIONARY ATOMS

% TIMBRE CLASSIFICATION OF A SINGLE MUSICAL INSTRUMENT

% RETRIEVAL OF PERCUSSION GESTURES USING TIMBRE CLASSIFICATION TECHNIQUES

% MUSICAL INSTRUMENT RECOGNITION IN POLYPHONIC AUDIO USING SOURCE-FILTER MODEL FOR SOUND SEPARATION

% KNOWLEDGE REPRESENTATION ISSUES IN MUSICAL INSTRUMENT ONTOLOGY DESIGN

% SPARSE CEPSTRAL AND PHASE CODES FOR GUITAR PLAYING TECHNIQUE CLASSIFICATION

% Ref Bittner MedleyDB
% Ref Dan Ellis typology of sound events
% Ref DCASE typology of acoustic events


Visipedia: \cite{belongie2015pattern}
Scattering transforms in musical instrument recognition: \cite{lostanlen2017}.

% For bibtex users:
\bibliography{ISMIRtemplate}

\end{document}
